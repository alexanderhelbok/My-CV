% !TeX spellcheck = de_AT
\documentclass{alex_cv}
\usepackage{setspace}

\setname{Alexander}{Helbok}
\setlocation{Innsbruck}{Austria} %City, Country
\setmobile{+43 681 81780037}
\setmail{alexander.helbok@hotmail.com}
\setthemecolor{alex} %you can play with color of the template (red is also nice..)

%\renewcommand{\baselinestretch}{2.5} 

\begin{document}
%	\fontsize{11pt}{10pt}\selectfont
	\setstretch{1.25}
	%Create header
	\headerview
	\vspace{2ex}\\
	%
	\begin{minipage}{\linewidth}
		{\large\scshape\bfseries\color{alex} Motivationsschreiben:}\\
		\normalsize Studienstiftungsseminare
	\end{minipage}
	%
	%Sections
	\section{}
%	Intro
	Mein Name ist Alexander Helbok und bin 21 Jahre alt. Derzeit befindet sich mein Wohnsitz in Innsbruck, wo ich im sechsten Semester Physik studiere. Ich komme aus dem ersten Jahrgang der Studienstiftung und schätze das Seminarangebot sehr, das mich bis nach Genf gebracht hat. Dieses Jahr freue ich mich besonders auf die Soiree, an der ich zum ersten Mal teilnehmen kann, und auf die vielfältigen Seminare. \\
	
%	Megacities
	Die Unterschiede zwischen Innsbruck (meinem derzeitigen Wohnort) und Wien (wo ich aufgewachsen bin) finde ich faszinierend.	Einerseits stellt die geografische Lage mit den Bergen (Platzmangel, Tourismus) ganz andere Anforderungen an die Stadt, was man auch an den Prioritäten der Stadtverwaltung sieht. Andererseits hat Wien die zehnfache Einwohnerzahl, wodurch die Infrastruktur ganz anders strukturiert und koordiniert werden muss. Der Sprung von der Millionenstadt eine Größenordnung nach unten habe ich ersterhands erlebt, wie es aber in die andere Richtung aussieht, würde ich mir gerne von Experten am Megacities Seminar zeigen lassen. Zudem ist es meines Erachtens nach sinnvoll, wenn das Megacities Seminar nicht voll mit Wienern ist ;) \\
	 
%	Bioarchäologie
	Ich studiere zwar Physik, habe aber das Archäologiestudium auch in Betracht gezogen (und tue das noch immer als Zweitstudium neben dem Master in Physik!). Ungelöste Rätsel ziehen mich generell an und die Frage, wo wir herkommen und wie unsere Ahnen gelebt haben finde ich wahnsinnig spannend. Mit Dinosauriern kenne ich mich dank meiner Kindheit recht gut aus, mit dem Werdegang des Menschen eher weniger, weshalb ich mich für das Archäologieseminar bewerben möchte. Unter anderem könnte es einen kleinen, aber sehr geschätzten Einblick ins Archäoligiestudium bieten. \\
	
%	Climate research
	Nach Innsbruck bin ich primär wegen der Natur gezogen. Die Berge bieten einen wunderschönen Lebensraum, in welchem man sich einerseits sportlerisch austoben kann, aber auch inmitten des Grünen seine Ruhe findet. Sosehr der Berg als Urgestein der Zeiten erscheint, so wird man leider des Öfteren in den Nachrichten erinnert, dass Nichts dem Klimawandel entkommt - auch nicht die schönen Berge Tirols. Aus meiner Leidenschaft für die Natur heraus motiviert, würde ich gerne tiefer in die Thematik eintauchen (um evtl. in Zukunft aktiv daran teilzuhaben), wo sich das Seminar Energiewende und Mobilität sehr gut eignet. 
	
	%Footnote
	\createfootnote
\end{document}
